\begin{table}[h!]
\scalefont{0.8}
\caption{Tabela tarefas do Gerente de Projetos e Configuração}
\label{tab1}
\begin{tabular}{|l|l|l|}
\hline
\begin{tabular}[c]{@{}l@{}}Item / \\ papel\end{tabular} & Gerente de Projetos & Aplicações \\ \hline
1 & Elaborar Termo de Abertura do Projeto & Formaliza o inicio do Projeto \\ \hline
2 & Monitoramento e Controle & Assegurar que objetivos sejam atingidos \\ \hline
3 & -Controle de Qualidade & Definir padrões em procedimentos, políticas e ações, de maneira uniforme. \\ \hline
4 & -Validar o Escopo & Processo de formalizar a aceitação das entregas do projeto. \\ \hline
5 & -Controlar o Escopo & Controlar as entregas do Projeto \\ \hline
6 & -Controlar os Riscos & \begin{tabular}[c]{@{}l@{}}Acompanhar os riscos identificados; Implementar os planos de respostas aos riscos; Monitorar os riscos \\residuais; Identificar novos riscos; Avaliar a eficácia do processo de riscos durante o ciclo de vida do projeto.\end{tabular} \\ \hline
7 & -Controlar o Cronograma & \begin{tabular}[c]{@{}l@{}}Atualizar o progresso do projeto, monitorar as variações entre o real com o planejado (linha de base) \\ e gerenciar as mudanças ocorridas.\end{tabular} \\ \hline
8 & -Controlar os Custos & Monitorar o status do projeto para atualizar o orçamento e gerenciar alterações na linha de base dos custos. \\ \hline
9 & Identificar Partes Interesadas & \begin{tabular}[c]{@{}l@{}}Processo de identificação de todas as pessoas ou organizações que podem ser afetadas pelo projeto e \\ documentação das informações relevantes relacionadas \\ aos seus interesses, envolvimento e impacto no sucesso do projeto.\end{tabular} \\ \hline
10 & Atualização  da TAP envio para o cliente & \begin{tabular}[c]{@{}l@{}}O termo de abertura do projeto deve conter informações sumarizadas porém com o nível de \\ detalhamento necessário para a aprovação ou não do projeto.\end{tabular} \\ \hline
11 & Encerramento do Projeto & \begin{tabular}[c]{@{}l@{}}Quando o projeto é encerrado o gerente de projeto deve finalizar todos relatórios, \\documentar a experiência do projeto, fornecer informação sobre o produto do projeto \\ e como atendeu seus requisitos do projeto, e então, documentar as lições aprendidas.\end{tabular} \\ \hline
12 & Escopo do Projeto & \begin{tabular}[c]{@{}l@{}}Trabalho que precisa ser realizado para entregar um produto, serviço ou resultado \\ com as características e funções especificadas\end{tabular} \\ \hline
13 & Elaboração do Plano de Gerenciamento de Versões & gerenciar diferentes versões no desenvolvimento do software e do documento \\ \hline
14 & Planejamento da Versão & \begin{tabular}[c]{@{}l@{}}planejar a versão do produto para a entrega, estabelecendo um plano e a meta que o \\Time e o resto da organização possam entender e comunicar.\end{tabular} \\ \hline
15 & Revisão Planejamento e Refinamento & Essa tarefa compreende o planejamento da homologação e a revisão geral do Plano de Projeto. \\ \hline
16 & Gerenciar Comunicação & \begin{tabular}[c]{@{}l@{}}é o processo de colocar as informações necessárias à disposição das partes interessadas no projeto \\ conforme planejado.\end{tabular} \\ \hline
17 & Mobilizar Equipes & tem como objetivo obter os recursos humanos necessários para o projeto. \\ \hline
18 & \begin{tabular}[c]{@{}l@{}}Formalizar \\ Encerramento\end{tabular} & \begin{tabular}[c]{@{}l@{}}Gerar o documento de encerramento, coletando assinatura das partes, \\ registrar pontos fortes e fracos do desenvolvimento.\end{tabular} \\ \hline
19 & \begin{tabular}[c]{@{}l@{}}Identificar Produtos\\ Reusaveis\end{tabular} & identificar e catalogar os produtos de software do projeto que podem ser reusados em outros projetos \\ \hline
20 & Registrar Base Histórica & \begin{tabular}[c]{@{}l@{}}conjunto de lições aprendidas deve ser compilado juntamente com os registros de\\ alterações da execução do projeto em relação ao planejamento.\end{tabular} \\ \hline
21 & Criar Baseline & \begin{tabular}[c]{@{}l@{}}conjunto de itens de configuração que, normalmente, representa o estado de desenvolvimento dos artefatos \\ do projeto em um determinado momento.\end{tabular} \\ \hline
22 & \begin{tabular}[c]{@{}l@{}}Adicionar Itens de \\ Configuração\end{tabular} & \begin{tabular}[c]{@{}l@{}}Os artefatos validados (novos ou alterados) que compõem a baseline devem ser inseridos \\ na base de itens de configuração.\end{tabular} \\ \hline
23 & Arquivar Baseline & arquivar a baseline anterior no Sistema de Gestão de Configuração. \\ \hline
24 & Liberar Baseline & disponibilizar a baseline para uso no Sistema de Gestão de Configuração. \\ \hline
\end{tabular}
\end{table}